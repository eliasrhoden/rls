\documentclass{article}
\usepackage[utf8]{inputenc}
\usepackage{amsmath}
\usepackage[margin=4cm]{geometry}
\setlength{\parindent}{0px}
\geometry{a4paper} 
\linespread{1.1} % Line spacing
\setlength{\parindent}{0px}
%\date{2018-11-27}
% FIGURES AND FLOATS
\usepackage{graphicx} % Required for including pictures
\usepackage{float} % Allows putting an [H] in \begin{figure} to specify the exact location of the figure
\usepackage{wrapfig} % Allows in-line images such as the example fish picture
\usepackage[font={small,it}]{caption}
\usepackage{subcaption}
\usepackage{epstopdf}
\graphicspath{{images/}}
\usepackage{comment}
\usepackage{gensymb}
\usepackage{textcomp}
\usepackage{enumitem}
% MATH
\usepackage{amssymb}
\usepackage{amsmath}
\usepackage{algorithm}
\usepackage[noend]{algpseudocode}
\usepackage{accents}

%\usepackage[options]{algorithm2e}

% OTHER
%\usepackage[]{mcode}
\usepackage{color}

\usepackage{booktabs}
\setlength\defaultaddspace{0.5ex}
\usepackage[math]{cellspace}
\setlength\cellspacetoplimit{3pt}
\setlength\cellspacebottomlimit{3pt}

% Cool Epsilon symbol
\newcommand{\ubar}[1]{\underaccent{\bar}{#1}}
\newcommand{\Epsilon}{\mathcal{E}}


\title{Recursive Least Squares}
\author{Elias R. (\textit{github.com/eliasrhoden})}

\begin{document}

\maketitle

\section{Recursive Least Squares}
From "Adaptive control" By K.J. Åström \& Björn Wittermark, the RLS algorithm is defined as:

\begin{gather}
K = P_{k-1} \, \varphi_{k-1} \, (\lambda + \varphi_{k-1}^T \, P_{k-1} \, \varphi_{k-1})^{-1} \\
P_k = (I - K\, \varphi_{k-1}^T) \, P_{k-1} \, \frac{1}{\lambda} \\
\theta_k = \theta_{k-1} + K \, (y_k - \varphi_{k-1}^T \, \theta_{k-1})
\end{gather}

For a sample intervall of $\delta_T$ and a desired forgetting time constant $T_f$ the choice of lambda is recomended as

\begin{equation}
\lambda = e^{-\delta_T/T_f}
\end{equation}



\end{document}
